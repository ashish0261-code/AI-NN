\documentclass[12pt]{article}

\begin{document}

\title{Moravec's paradox}
\author{Ashish Saini (18111013)}
\date{July 28, 2021}
\maketitle



\section{Summary}

Moravec’s paradox is a phenomenon surrounding the abilities of AI-powered tools. It observes that tasks humans find complex are easy to teach AI. Compared, that is, to simple, sensorimotor skills that come instinctively to humans.  \par

In the 1980s, Hans Moravec, Rodney Brooks, Marvin Minsky and others articulated and discussed this AI paradox. As Moravec put it:

 \begin{center}
“It is comparatively easy to make computers exhibit adult level performance and difficult or impossible to give them the skills of a one-year-old.”
 \end{center}

For example, artificial intelligence can complete tricky logical problems and advanced mathematics. But the ‘simple’ skills and abilities we learn as babies and toddlers — perception, speech, movement, etc. — require far more computation for an AI to replicate. In other words, for AI the complex is easy, and the easy is complex.   \par

For a start, the skills that we define as ‘simple’ — those we learn instinctively — are products of years and years of evolution. So, while they may appear simple, it’s only because of billions of years’ worth of tuning. In other words, the complexity of the simple abilities we take for granted is invisible. Plus, AI ‘learns’ through us telling it how to do things. We’ve consciously learned how to do mathematics, win games and follow logic. We know the steps (computations) needed to complete these tasks. And so, we can teach them to AI. But how do you tell anything how to see, hear, or move? We don’t consciously know all the computations needed to complete these tasks. These skills are not broken down into logical steps to feed into an AI. As such, teaching them to an AI is extremely difficult.    \par

The history of AI has seen an impact from Moravec’s paradox. In fact, it’s arguably a factor that held back development and contributed to the AI effect. The AI effect is a phenomenon that has seen AI-powered tools lose their ‘AI’ label over time, due to not being ‘true’ intelligence. Moravec’s paradox could have contributed to this. That is, the reason these tools lost their ‘intelligent’ status is that the tasks it does are simple, once you break them down.




\end{document}