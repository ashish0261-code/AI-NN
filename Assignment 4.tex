\documentclass[12pt]{article}

\begin{document}

\title{Physical Sciences}
\author{Ashish Saini (18111013)}
\date{August 06, 2021}
\maketitle



\section{Summary}

Physical Science, the systematic study of the inorganic world, as distinct from the study of the organic world, which is the province of biological science. Physical science is ordinarily thought of as consisting of four broad areas: astronomy, physics, chemistry, and the Earth sciences. Each of these is in turn divided into fields and subfields. This article discusses the historical development—with due attention to the scope, principal concerns, and methods—of the first three of these areas.\par

Physics, in its modern sense, was founded in the mid-19th century as a synthesis of several older sciences—namely, those of mechanics, optics, acoustics, electricity, magnetism, heat, and the physical properties of matter. The synthesis was based in large part on the recognition that the different forces of nature are related and are, in fact, interconvertible because they are forms of energy.

 \par

The boundary between physics and chemistry is somewhat arbitrary. As it developed in the 20th century, physics is concerned with the structure and behaviour of individual atoms and their components, while chemistry deals with the properties and reactions of molecules. These latter depend on energy, especially heat, as well as on atoms; hence, there is a strong link between physics and chemistry. Chemists tend to be more interested in the specific properties of different elements and compounds, whereas physicists are concerned with general properties shared by all matter.  
\par

Astronomy is the science of the entire universe beyond Earth; it includes Earth’s gross physical properties, such as its mass and rotation, insofar as they interact with other bodies in the solar system. Until the 18th century, astronomers were concerned primarily with the Sun, Moon, planets, and comets. During the following centuries, however, the study of stars, galaxies, nebulas, and the interstellar medium became increasingly important. Celestial mechanics, the science of the motion of planets and other solid objects within the solar system, was the first testing ground for Newton’s laws of motion and thereby helped to establish the fundamental principles of classical (that is, pre-20th-century) physics. \par

\begin{center}
"AI can be used in psuedoscience by identifying Nuances in Fake news vs. Satire using semantic and linguistic cues."
\end{center}





\end{document}