\documentclass[12pt]{article}

\begin{document}

\title{Gödel's Incompleteness Theorems}
\author{Ashish Saini (18111013)}
\date{July 21, 2021}
\maketitle



\section{Summary}

Gödel's incompleteness theorems are two theorems of mathematical logic that are concerned with the limits of provability in formal axiomatic theories. These results are important both in mathematical logic and in the philosophy of mathematics. The theorems are widely, but not universally, interpreted as showing that Hilbert's program to find a complete and consistent set of axioms for all mathematics is impossible.

\begin{itemize}

\item The first incompleteness theorem states that no consistent system of axioms whose theorems can be listed by an effective procedure (i.e., an algorithm) is capable of proving all truths about the arithmetic of natural numbers. For any such consistent formal system, there will always be statements about natural numbers that are true, but that are unprovable within the system.
\item The second incompleteness theorem, an extension of the first, shows that the system cannot demonstrate its own consistency.

\end{itemize}

Employing a diagonal argument, Gödel's incompleteness theorems were the first of several closely related theorems on the limitations of formal systems. They were followed by Tarski's undefinability theorem on the formal undefinability of truth, Church's proof that Hilbert's Entscheidungsproblem is unsolvable, and Turing's theorem that there is no algorithm to solve the halting problem.   \par

The incompleteness theorems apply to formal systems that are of sufficient complexity to express the basic arithmetic of the natural numbers and which are consistent, and effectively axiomatized, these concepts being detailed below. Particularly in the context of first-order logic, formal systems are also called formal theories. In general, a formal system is a deductive apparatus that consists of a particular set of axioms along with rules of symbolic manipulation (or rules of inference) that allow for the derivation of new theorems from the axioms. One example of such a system is first-order Peano arithmetic, a system in which all variables are intended to denote natural numbers. In other systems, such as set theory, only some sentences of the formal system express statements about the natural numbers. The incompleteness theorems are about formal provability within these systems, rather than about "provability" in an informal sense.


\end{document}